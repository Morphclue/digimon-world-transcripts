\section{Transkript - Person 3}
\sloppy
\texttt{\begin{itemize}[]
    \setlength\itemsep{0.02em}
    \linenumbers
    \item \interview{IV} Wie alt bist du?
    \item \interview{P3} 24.
    \item \interview{IV} Welchem Geschlecht fühlst du dich zugehörig?
    \item \interview{P3} Männlich.
    \item \interview{IV} Fühlst du dich vorbereitet?
    \item \interview{P3} Ja.
    \item \interview{IV} Welche Spiele hast du in letzter Zeit in der Freizeit gespielt?
    \item \interview{P3} Genshin Impact und allgemein ein paar Indie-Spiele. 
    \item \interview{IV} Welche wären das zum Beispiel?
    \item \interview{P3} Forager oder Risk of Rain 2.
    \item \interview{IV} Spielst du Spiele lieber mit Maus und Tastatur oder mit einem Controller?
    \item \interview{P3} Das ist spielabhängig. Wenn ein Spiel aber eher entspannt sein soll, dann spiele ich das lieber mit Controller. 
    \item \interview{IV} Nutzt du lieber eine Konsole oder einen Computer zum spielen?
    \item \interview{P3} Ich nutze einen PC lieber. 
    \item \interview{IV} Hattest du damals ein Tamagotchi oder etwas Vergleichbares besessen?
    \item \interview{P3} Ja. Ich hatte damals einen Tamagotchi.
    \item \interview{IV} Ist dir das Digimon-Franchise bekannt?
    \item \interview{P3} Ja. Ich habe damals immer vereinzelt Folgen im Fernsehen gesehen. Allerdings habe ich die Serie nie durchgeschaut. Aber die Serie hat mich auch nie wirklich stark interessiert. Allerdings hat mich Digimon World als Spiel sehr begeistert. 
    \item {-------------------} Spiel beginnt {-------------------}
    \item \interview{DW} \frq Hast du ein Digivice?\flq{}
    \item \interview{P3} Ja. 
    \item \interview{DW} \frq Nächste Frage. Was magst du lieber: Tag oder Nacht?\flq{} 
    \item \interview{P3} An dieser Stelle ist auch schon die erste Entscheidung die bestimmt welchen Partner man am Anfang erhält. Ich selbst präferiere Agumon, weil er einen Angriff hat der im frühen Spielverlauf sehr stark ist und auf Tempo skaliert.
    \item \interview{P3} An dieser Stelle kann man sich Fleisch auf der Fleischfarm abholen. Das würde für einige vielleicht keinen Sinn machen, aber das hier ist die Digimon Welt. Hier ist theoretisch alles möglich. 
    \item \interview{P3} Jetzt gehe ich erst einmal noch zu Jijimons Haus. Hier kriegt man am Anfang direkt ein paar Items. Eine Art Willkommenspaket. 
    \item \interview{P3} Wir gehen auch direkt einmal ins grüne Trainingscamp, weil gleich auch einer der ersten Kämpfe stattfindet. Das ganze mache ich bis das Digimon zum ersten Mal auf die Toilette muss. Man sieht auch schon, dass das Digimon Hunger hat. Allerdings kann man das ruhig einmal auslassen. 
    \item \interview{P3} Jetzt müsste ich nur noch ein oder zwei Mal trainieren, dass er aufs Klo muss. 
    \item \interview{DW} (Digimon symbolisiert, dass es auf Klo muss)
    \item \interview{P3} Es ist generell auch so, dass ich den Anfang sehr oft gespielt habe. Ich weiß gar nicht wie viele Spielstände ich offiziell habe. Ich habe das Spiel aber offiziell noch nicht zu 100\% durchgespielt. Wie man hier in den einzelnen Menüs sehen kann wäre das auch sehr zeitaufwendig. Die Dinge die man mit Geld oder Karten sammeln kann habe ich meistens. Allerdings sind die Turniere oftmals das was mir fehlt. \item \interview{IV} Du erwähnst gerade, dass du vieles bereits im Spiel gesammelt hast. Zählt dazu auch, dass du alle Dorfbewohner in die Stadt geholt hast?
    \item \interview{P3} Ja. Ich habe es bereits geschafft alle Digimon in die Stadt zu holen. Ebenfalls habe ich auch fast alle Evolutionen für die Digimon geschafft. Nur irgendwann war dann auch erst einmal die Luft raus aus dem Spiel. Vor allem, wenn man dann auch direkt ein neues Digimon züchten muss, dann frisst das auch viel Zeit. Da wäre natürlich das Thema Emulation ziemlich interessant, weil man dort die Zeit schneller einstellen könnte. Ich habe bis jetzt selbst noch nichts emuliert, aber man sieht das ja immer wieder im Internet.
    \item \interview{IV} Jetzt wo du die Funktion generell ansprichst mit der Zeiteinstellung. Hättest du gerne einen Knopf auf dem Controller mit dem du das Spiel schneller machen könntest?
    \item \interview{P3} Im Grunde genommen ja, weil es auch zeitaufwändig ist. Allerdings nimmt das ganze auch die Atmosphäre vom Spiel weg. Also es gibt manche Kämpfe die können sich in die Ewigkeit ziehen. Da macht man ein bis zwei Schaden pro Angriff und der Gegner hat gefühlt zehntausend Leben. 
    \item \interview{P3} Da wurde aber relativ gut vom Spiel auch versucht entgegenzuwirken. Nach einiger Zeit erhält man seinen ultimativen Angriff und der durchdringt die Rüstung des Gegners. Das bedeutet, selbst, wenn dieser Buffs benutzt, dann werden diese vom Angriff ignoriert. Hängt natürlich auch davon ab wie schnell man das ganze beim ultimativen Angriff drückt. 
    \item \interview{DW} (Der erste Kampf beginnt)
    \item \interview{P3} Hier sieht man schon das beste Beispiel für Momente wo sich beide nur gegenseitig anstarren und nichts passiert. Ich weiß nur, dass wenn man ungefähr 150 Tempo hat pro zehn Sekunden 2 Angriffe machen kann. 
    \item \interview{IV} Du hattest vorhin erwähnt, dass man beim ultimativen Angriff die Schultertasten schnell drücken kann. Wie hast du damals herausgefunden, dass das funktioniert?
    \item \interview{P3} Das habe ich tatsächlich erst durch meine Cousins herausgefunden, dass es so etwas gibt. Ich habe mich vorher aber natürlich schon gefragt wieso da eine Leiste plötzlich kommt und man nichts drücken kann. Da habe ich als Kind nur versucht die Tasten zu drücken die ich direkt vor mir gesehen habe. Aber ich hatte nie die Schultertasten im Blick.
    \item \interview{IV} Fändest du es besser, wenn angezeigt wäre, welche Knöpfe gedrückt werden müssen?
    \item \interview{P3} Das auf alle Fälle. So eine Anzeige wie in manchen Spielen wäre super. 
    \item \interview{P3} Das ist halt generell so bei Digimon World, dass man viele Dinge selbst herausfinden muss. Also ich sag auch selber, dass, obwohl ich so viel gespielt habe, immer noch neue Dinge herausfinde. Aber das liegt auch daran, dass ich mich nicht im Internet spoiler. 
    \item \interview{IV} Findest du den Zeitrahmen für den ultimativen Angriff angemessen?
    \item \interview{P3} Ich finde den fair, aber es schädigt dem Controller da so schnelles es geht auf die Tasten hauen zu müssen. 
    \item \interview{P3} Ich bleibe jetzt an dieser Stelle stehen, bis der Nachmittag angebrochen ist.
    \item \interview{DW} (Es ist Nachmittag)
    \item \interview{P3} Jetzt ist es Nachmittag. Das sieht man zum einen an der Uhr, aber auch an dem Licht im Spiel. Wieso das so wichtig ist? Weil hier ein Digimon ist, welches nur am Nachmittag erscheint. Der wartet im Wasser und bringt uns direkt zur nächsten Insel. 
    \item \interview{IV} Wie war das damals mit der Uhr? Hattest du Schwierigkeiten die zu lesen?
    \item \interview{P3} Ich habe erst vor kurzem angefangen die Uhr in Digimon World lesen zu lernen. Ich habe mich nämlich öfters davon verwirrt gefühlt. Es gibt einige zeitabhängige Kämpfe und ich habe die manchmal verpasst, weil ich mich an eine 24-Stunden Uhr orientiert habe. 
    \item \interview{IV} Aus welchem Grund dachtest du, dass es sich um keine 12-Stunden Uhr handelt?
    \item \interview{P3} Weil die Uhr nur einen Zeiger hat. Allerdings hatte ich danach verstanden, dass der gelbe Punkt einen weiteren darstellen sollte. 
    \item \interview{P3} Ich hatte noch etwas vergessen zu erwähnen. Wir haben ja bereits mehrere Digimon in die Stadt geholt. Agumon hat jetzt die Bank eröffnet, Palmon hat die Fleischfarm ausgebaut und wir haben jetzt Betamon in die Stadt geholt. Sprich es gibt größere Steaks und das Digimon ist schneller satt.  Betamon ist meiner Meinung nach einer der wichtigsten, weil er den Shop in der Stadt öffnet und der beinhaltet viele verschiedene Items. 
    \item \interview{P3} Ich finde in dem Spiel auch ziemlich cool, dass man nur speichert wenn man im Spiel schläft. Als Kind habe ich mir damals immer die Frage am Ende des Tages gestellt: War der Tag wirklich produktiv, dass ich jetzt speichern sollte oder habe ich Fehler gemacht und versuche ich das ganze noch einmal? Also der typische Perfektionismus. Meinem Digimon muss es gut gehen und er darf keine Pflegefehler erhalten. 
    \item \interview{IV} Jetzt wo du bereits erwähnst, dass du wolltest, dass es deinem Digimon gut geht. Wie genau siehst du die emotionale Bindung zu deinem Partner?
    \item \interview{P3} Also damals als Kind würde ich da definitiv ja sagen. Ich saß vor dem Fernseher und war auch traurig als mein Digimon gestorben ist. Jetzt bin ich natürlich nicht mehr so emotional von so etwas berührt, aber ich würde sagen, dass das ganze noch von ein paar Faktoren abhängt. Wenn man sein Digimon von der kleinsten Evolutionsstufe bis zur höchsten gebracht hat und man stolz darauf ist, dass es funktioniert hat, dann würde ich auch ja sagen. 
    \item \interview{IV} Also würdest du als Fazit für die Frage sagen, dass es ganz von der Zeit abhängt die du mit deinem Partner verbracht hast?
    \item \interview{P3} Genau. Abgesehen vom ersten Digimon. Bei dem Digimon weiß ich bereits wie ich das trainieren möchte und wie es sich entwickeln wird. Beim zweiten Digimon mache ich meistens noch die restlichen Sachen und ab dem dritten oder vierten plane ich zwar welches ich erreichen möchte, aber das klappt nicht immer. Falls das aber klappt ist das schon eher ein Erfolgserlebnis. Ich kenne zum Beispiel auch nicht von jedem Digimon die Animation wie er sich freut oder ähnliches. 
    \item \interview{IV} Also ist das ganze für dich auch noch dieser Erkundungsdrang der dich antreibt?
    \item \interview{P3} Genau, deswegen spoiler ich mich auch ungern. Außer es zerreißt mich wirklich, weil ich keinen Ansatz habe. 
    \item \interview{DW} (Der Kampf mit Kunemon beginnt)
    \item \interview{P3} Hier sieht man auch ein Problem, welches manchmal nervig ist. Ich kann mein Digimon nicht ganz erkennen und weiß nicht wann es wieder angreifen kann oder ich den Knopf für den ultimativen Angriff drücken kann, weil das HUD nicht ganz sichtbar ist aufgrund der fixen Kamera. 
    \item \interview{IV} Gibt es generell Dinge die dich an der Oberfläche noch stören?
    \item \interview{P3} Nicht wirklich stören, aber es gab am Anfang ein Verständnisproblem bei den unteren Leisten. Ich habe einige Zeit gebraucht bis ich verstanden habe, dass diese sich zwei mal füllen müssen. Das war auch so, dass ich zuerst auch die Auswirkungen auf den Partner nicht kannte. Zum Beispiel, dass man ihn auch schlecht behandeln muss für bestimmte Entwicklungen wusste ich nicht.
    \item \interview{IV} Würdest du generell sagen, dass du die Option behalten möchtest ihn schlecht behandeln zu können?
    \item \interview{P3} Es gibt zwei Digimon die man nicht mag. Das weiß glaube ich auch jeder der das Spiel bereits einmal gespielt hat. Die beiden kann man ruhig schlecht behandeln.
    \item \interview{IV} Du hattest vorhin erwähnt, dass du weißt welche Werte du trainieren musst, damit Digimon sich in bestimmte Digimon weiter entwickeln. Welche Werte hast du damals trainiert als du das noch nicht genau wusstest?
    \item \interview{P3} Ich habe damals fast nur Leben trainiert, weil ich nicht wusste, dass man Items im Kampf verwenden kann. 
    \item \interview{IV} Hättest du dir gewünscht, dass es auch hier eine Anzeige gäbe, dass man sein Inventar mit einem bestimmten Knopf öffnen kann?
    \item \interview{P3} Ja, das wäre viel besser gewesen. 
    \item \interview{P3} Ich wusste damals auch nicht, dass man das Digimon tadeln kann, damit er etwas isst was er vorher verweigert hat. 
    \item \interview{IV} Wie findest du das generell, dass man sein Digimon dafür tadeln muss?
    \item \interview{P3} Ich finde es tatsächlich jetzt belohnend, weil er dadurch Disziplin erhält ohne andere Werte zu verlieren. Aber als Kind hat diese Information leider gefehlt. Also wieder das Thema, dass Informationen fehlen. 
    \item \interview{IV} Woher weißt du was du genau aktuell alles zu tun hast? 
    \item \interview{P3} Ich verfolge ein bestimmtes Muster und hab von Anfang an schon einen groben Plan für die erste Tage ausgelegt. Also wie vorher erwähnt, damit auch die wichtigsten Sachen in der Stadt ausgebaut werden. 
    \item \interview{DW} (Das Partner-Digimon stirbt im Kampf)
    \item \interview{P3} Das war jetzt nicht geplant, aber ich setze das Spiel mal kurz zurück.
    \item \interview{IV} Vielleicht noch eine Frage während das Spiel gerade startet. Findest du, dass Digimon World immersiv ist? Also hast du vielleicht auch das Gefühl, dass du die Zeit um dich herum vergisst?
    \item \interview{P3} Damals als Kind schon. Also da gibt es auch eine lustige Geschichte zu. Meine Eltern haben damals das neue Haus renoviert und wegen meiner Stauballergie konnte ich nicht viel helfen. Also habe ich mich morgens hingesetzt, Digimon World gespielt und dann hatten wir irgendwie schon Abends. Es war ungefähr 18 oder 19 Uhr und wir sind wieder zurück in die alte Wohnung gegangen. Also die Zeit ging schon ziemlich schnell vorbei als man sich vertieft hatte. Normalerweise brauche ich für die Routine am Anfang des Spiels zwei bis drei Stunden. Ich würde sagen, dass man beim farmen von Sachen das ganze nur passiv macht. Also man konzentriert sich nicht hundertprozentig auf das Spiel. Auch beim Entwickeln vom kleinsten bis zum größten Level ist das eher so ein passives Ding. Das Training nimmt dann ungefähr 20 bis 30 Minunten in Anspruch. Man muss eigentlich nur eine Taste drücken und ab und zu schauen, dass er etwas zu essen bekommt oder auf die Toilette gehen kann. 
    \item \interview{IV} Fändest du es besser, wenn man diese Pflegebedingungen dauerhaft angezeigt bekommt oder findest du die aktuelle Implementierung besser?
    \item \interview{P3} Was meiner Meinung nach fehlt ist, dass das ganze nicht in einem Menü angezeigt wird. Sprich, wenn ich das Training beginnen möchte sehe ich das ganze meist zu spät angezeigt. Wenn ich dann das Training anfange und dadurch Zeit vergeht ist das ganze meist zu spät. Dadurch kriegt man dann einen weiteren Pflegefehler und das will man natürlich vermeiden. Was ich gut fände wäre, wenn man davor kurz gewarnt wird. Mit einer Nachricht wie \frq Hey, dein Digimon hat noch Ansprüche, welche du erfüllen musst bevor du mit dem Training beginnst\flq{}. Man könnte das ganze neben den Sprechblasen auch als Status-Symbole neben den anderen Leisten in der Oberfläche machen. Ähnlich wie man das auch in Pokémon hat, wenn Pokémon vergiftet, paralysiert oder anderes als Status besitzen. 
    \item \interview{IV} Findest du, dass die Kämpfe lohnenswert sind? Also abgesehen von denen die in die Stadt ziehen natürlich.
    \item \interview{P3} Also der den ich gerade bekämpft habe war lohnenswert, weil er ein Item fallen gelassen hat.
    \item \interview{IV} Wenn du jetzt die Wahl zwischen Training und Kampf hättest, welches würdest du wählen? 
    \item \interview{P3} Training. Das ist effizienter für den Start und auch später im Spiel. Kämpfen ist in dem Spiel nur wichtig, um neue Angriffe zu lernen.
    \item \interview{IV} Was denkst du generell von den Angriffen und wie genau hast du diese ausgerüstet?
    \item \interview{P3} Man kann drei Angriffe ausrüsten und damals dachte ich, dass man das auch machen sollte. Mittlerweile weiß ich natürlich, dass es besser ist nur einen Angriff auszuwählen. Das macht man, damit das Digimon diesen auch immer macht und nicht variiert. Oftmals sind Angriffe stark die die gegnerische Kampf-Animation unterbrechen. \item \interview{P3} In der Stadt gibt es verschiedene Bewohner die Informationen zu verschiedenen Sachen im Spiel geben. Der hier zum Beispiel erzählt zwar, dass es verschiedene Digitationsstufen gibt, aber erklärt nicht wie man diese erreichen kann. Er gibt also eine Art Grundverständnis davon. Das sollte tatsächlich aber auch im Handbuch von Digimon World stehen. Als Kind war dieses Handbuch allgemein sehr wichtig. Man hat dort verschiedene Orte gesehen die man nicht kannte und hat einen Weg dahin gesucht. Leider konnte man nicht alle erreichen, weil man die deutsche Version von Digimon World nicht durchspielen konnte. 
    \item \interview{P3} Ich trainiere jetzt noch ein wenig Tempo und achte darauf nicht auf mehr als 150 Tempo zu kommen, weil ich sonst Greymon nicht mehr bekomme.
    \item \interview{P3} Du hattest vorher auch etwas wegen dem Tadeln gefragt. Da hätte ich jetzt vielleicht noch eine Anmerkung. 
    \item \interview{IV} Gerne.
    \item \interview{P3} Damals war es auch so, dass man sich aufgeregt hatte, wenn ein Digimon sich nicht wie in der Serie weiterentwickelt hat. Normalerweise ist es ja so, dass Agumon sich in der Serie zu Greymon weiterentwickelt. Allerdings hat man hier im Spiel manchmal ein Numemon erhalten und war dann natürlich irgendwie enttäuscht. Da habe ich dann manchmal getadelt. Aber mit mehr Spielzeit kam dann auch die Erkenntnis, dass Digimon sich auch in andere Digimon weiterentwickeln können. 
    \item \interview{P3} Also ich würde allgemein sagen, dass man mit viel Vorbereitung und Erfahrung Kämpfe in Digimon World gut bestreiten kann. Oftmals ist es so, dass einige Heildisks schon komplett ausreichen. Gegen Ende hin gibt es nur einige Kämpfe die etwas komplizierter werden. Aber nur, weil man einen Gegner drei Mal bekämpfen muss oder er extrem viel Leben besitzt. 
    \item \interview{IV} Wie gehst du die meisten Aktionen im Spiel an?
    \item \interview{P3} Über Trial and Error\footnote{Eine Methode bei der man ein Problem so lange mit verschiedenen Möglichkeiten angeht bis dieses gelöst wird.}. Wir sind hier ja gerade bei Centauromon und da wusste ich damals auch nicht so ganz wie man da vorbeikommen sollte. Als Kind habe ich mir die Karte die vor dem Eingang steht abgezeichnet und versucht den Weg nachzulaufen. Mittlerweile habe ich aber verstanden wie das funktioniert.
    \item \interview{IV} Was ist aktuell dein Plan?
    \item \interview{P3} Ich würde gerne noch mein Digimon digitieren lassen. Allerdings sind die Voraussetzungen für eine Digitation bei Greymon und Centauromon ziemlich ähnlich. Ich hoffe aber, dass ich das Greymon erhalte. Ich plane dann erst einmal die Höhle fertig zu machen. Danach würde ich in Richtung Birdramon gehen und die Brücke überqueren. Aber das ist natürlich alles wieder einkalkuliert. Ich mache das alles, weil es für mich die größte Effizienz hat. Ich habe mir auch mehrere Speedruns\footnote{Ein Speedrun ist die Bezeichnung für das Durchspielen eines Spiels in schnellstmöglicher Zeit.} zu dem Spiel angeschaut und da schaut man sich dann natürlich ein paar Sachen ab.
    \item \interview{P3} (Hilft im Tunnel aus)
    \item \interview{P3} Beim Helfen im Tunnel ist es aber wieder das Problem, dass teilweise Informationen fehlen. Dass Digimon sagt zwar, dass durch das Helfen der Fortschritt der Bohrungen vorangeht, allerdings nicht wie schnell. Ich fände es gut, wenn er sagen würde wie lange das ganze dauert wenn ich ihm aushelfe und wie lange wenn ich nicht aushelfen würde. 
    \item \interview{IV} Aus welchem Grund tadelst du dein Digimon gerade?
    \item \interview{P3} Damit die Disziplin hoch geht.
    \item \interview{IV} Findest du, dass das förderlich für das Verhältnis zwischen Partner und dir ist?
    \item \interview{P3} Nein. Aber damit es zu Greymon wird muss ich das leider machen, damit die Disziplin für die Digitation ausreicht. 
    \item \interview{IV} Wie hast du damals herausgefunden, welche Digimon in die Stadt kommen und welche nicht?
    \item \interview{P3} Man hat die an der Farbe erkannt, weil man die zum Beispiel aus dem Anime kannte.
    \item \interview{P3} Jetzt kommt ein Minispiel bei dem ich gerne vorher schlafen gehe, damit das Spiel speichert. Ich starte das Spiel meistens neu, weil das ganze irgendwie ziemlich zufällig ist.
    \item \interview{IV} Aus welchem Grund startest du das Spiel neu und probierst das Minispiel nicht erneut?
    \item \interview{P3} Weil mir die Zeit die dabei verloren gehen würde zu kostbar ist.
    \item \interview{DW} (Digimon digitiert)
    \item \interview{P3} Super cool. Ein Greymon. Genau wie geplant. Es ist aber immer noch ein Moment der mich glücklich macht, wenn das auch wirklich klappt.
    \item \interview{IV} Du bist ja gerade auf dem Weg Birdramon in die Stadt zu holen. Wie findest du das schnelle Reisen in Digimon World, welches durch Birdramon freigeschaltet wird?
    \item \interview{P3} Ich finde es zu teuer. Wenn es günstiger wäre, dann würde man das vermutlich mehr nutzen, aber so ist es einfach zu teuer. Sonst ist es natürlich gut, dass man schnell an Orte kommt. Aber allgemein ist es gar nicht so schwierig manche Orte zu Fuß zu erreichen.
    \item \interview{IV} Was hältst du von der Idee, dass man den Digitationsgraphen bei einer Entwicklung in den Vordergrund rückt? Würde das deiner Meinung nach mehr Klarheit schaffen?
    \item \interview{P3} Das auf alle Fälle. Wenn man das dann noch deutlich macht welches man aktuell besitzt und welches man bekommt wäre das nicht verkehrt. 
    \item {-------------------} Abschlussfragen {-------------------}
    \item \interview{IV} Was empfindest du als gut in Digimon World?
    \item \interview{P3} Das hatte ich ja vorher schon einige Male angesprochen. Für mich ist der Erkundungsfaktor ziemlich spannend und interessant. Also mit limitierten Informationen möglichst viel erkunden. Das ganze ist vergleichbar mit einem Mythos den man lüften möchte oder einem Gerücht. Sonst würde ich natürlich auch sagen, dass das ganze für mich ein Kindheitsspiel ist. Also ich gehe da an vielen Stellen noch mit viel Nostalgie dran. Ich habe das Spiel damals als Kind auch zunächst nicht ausgewählt, weil es ein Tamagotchi-Spiel ist, sondern eher, weil es Digimon war und ich das bereits aus der TV-Serie kannte.
    \item \interview{IV} Du hast gerade erwähnt, dass du das nicht primär wegen dem Tamagotchi-System ausgewählt hast. Ist das denn trotzdem ein Pluspunkt als du herausgefunden hast, dass das Spiel dieses Feature beinhaltet? Also wie findest du das Genre?
    \item \interview{P3} Also ich würde sagen, dass das Tamagotchi-Feature an sich auch mal lästig werden konnte. Das ständige auf die Toilette bringen oder füttern konnte auch anstrengend sein. Aber man gewöhnt sich auch irgendwann daran und stellt sich auf die Zeiten etwas besser ein.
    \item \interview{IV} Was findest du nicht so gut in Digimon World umgesetzt?
    \item \interview{P3} Ich würde sagen, dass man bei den Items etwas mehr Informationen geben könnte. Wo wir natürlich auch wieder beim Thema mangelnde Informationen sind. Manchmal dauert es ziemlich lange bis man in der Story des Spiels fortfahren kann. Das liegt dann meistens an Faktoren, dass man zum Beispiel das Digimon wieder aufs Neue trainieren muss. Also allgemein dauert das ganze grinden\footnote{Grinding beschreibt das wiederholte Ausführen von Aufgaben in Videospielen, um ein bestimmtes Ziel zu erreichen.} ziemlich lange. Die ersten Entwicklungsstufen sind dann meistens gar nicht erst kampffähig, weswegen man dazu regelrecht gezwungen wird.
    \item \interview{IV} Würdest du sagen, dass man diese auch weglassen könnte oder vielleicht das Anfangsgebiet mit leichteren Gegnern ausstatten könnte?
    \item \interview{P3} Ja, würde ich so sagen. Beide Lösungen klingen plausibel. Man traut sich mit einem Baby-Digimon einfach nicht raus und auch, wenn man in einem Kampf verwickelt wird merkt man, dass man einfach gar keinen Schaden macht.
    \item \interview{IV} Wo hatte das Kampfsystem deiner Meinung nach Stärken?
    \item \interview{P3} Der Zufallsfaktor im Spiel ist ganz cool und auch, dass man nicht zu viel Einfluss hat. 
    \item \interview{IV} Was fandest du nicht so gut realisiert?
    \item \interview{P3} Ich hätte gerne ein klein bisschen mehr Kontrolle im Kampf oder mehr Optionen mich vor dem Kampf vorzubereiten. Man könnte das ganze auch so realisieren, dass man von Anfang an entscheiden kann, welchen Angriff das Digimon wählt. Allerdings wäre hier die Idee, dass er eine prozentuale Chance hat davon abzuweichen. Man müsste sich auch die Frage stellen ob man das ganze echtzeit oder rundenbasiert macht. 
\end{itemize}}
\nolinenumbers