\section{Transkript - Person 2}
\sloppy
\texttt{\begin{itemize}[]
    \setlength\itemsep{0.02em}
    \linenumbers
    \item \interview{IV} Wie alt bist du?
    \item \interview{P2} 21.
    \item \interview{IV} Welchem Geschlecht fühlst du dich zugehörig?
    \item \interview{P2} Weiblich.
    \item \interview{IV} Fühlst du dich für das Interview ausreichend vorbereitet?
    \item \interview{P2} Ja.
    \item \interview{IV} Welche Spiele hast du in letzter Zeit in der Freizeit gespielt?
    \item \interview{P2} League of Legends, Teamfight Tacticts, Dead by Daylight und Genshin Impact.
    \item \interview{IV} Welche Genres spielst du allgemein gerne?
    \item \interview{P2} Ich würde sagen, dass ich entspannte Genres spiele.
    \item \interview{IV} Spielst du Spiele lieber mit Maus und Tastatur oder mit einem Controller?
    \item \interview{P2} Am PC mit Maus und Tastatur. 
    \item \interview{IV} Nutzt du lieber eine Konsole oder einen Computer zum spielen? 
    \item \interview{P2} Ich spiele lieber am PC. 
    \item \interview{IV} Hattest du damals ein Tamagotchi oder etwas Vergleichbares?
    \item \interview{P2} Nein. Ich habe nur Sims gespielt, falls das dazu zählen würde.
    \item \interview{IV} Ist dir das Digimon-Franchise bekannt?
    \item \interview{P2} Nein. Ich würde sagen, dass ich den Namen schon einmal gehört habe, aber mehr auch nicht.
    \item {-------------------} Spiel beginnt {-------------------}
    \item \interview{DW} \frq Hast du ein Digivice?\flq{}
    \item \interview{P2} Nein. Ich weiß nicht einmal was das ist.
    \item \interview{DW} \frq Nächste Frage. Was magst du lieber: Tag oder Nacht?\flq{} 
    \item \interview{P2} Nacht.
    \item \interview{P2} Oh nein.
    \item \interview{IV} Was ist denn passiert?
    \item \interview{P2} Mein Name hat zu viele Zeichen, um dargestellt zu werden.
    \item \interview{IV} Und was planst du jetzt zu tun?
    \item \interview{P2} Ich werde glaube ich eine kürzere Form von meinem Namen nehmen müssen. Dabei mag ich verkürzte Namen überhaupt nicht. Ich finde das auch ziemlich unangenehm, wenn andere sich gegenseitig so nennen.
    \item \interview{P2} Ich kann mir vorstellen, dass die Zielgruppe jüngere Personen sind. 
    \item \interview{IV} Wie kommst du auf diesen Gedanken?
    \item \interview{P2} Gute Frage. Die Sequenz am Anfang lässt mich das stark denken.
    \item \interview{DW} (Hauptcharakter befindet sich in der Mitte von mehreren Digimon)
    \item \interview{P2} Kann ich mir jetzt ein Digimon aussuchen? Das wäre cool.
    \item \interview{P2} Moment mal. Ich finde es auch doof, dass ich gar kein Mädchen spielen kann.
    \item \interview{IV} Aus welchem Grund würdest du das lieber wollen?
    \item \interview{P2} Ich identifiziere mich mit diesem Charakter nicht so sehr. Ich muss schon sagen, dass mich das ziemlich stört.
    \item \interview{IV} Kannst du noch näher erläutern aus welchem Grund das so ist?
    \item \interview{P2} Ich habe das Gefühl, dass das Spiel nicht für mich gemacht ist. Also es ist etwas für Jungs und ich gehöre nicht dazu.
    \item \interview{IV} Wenn das Spiel unterschiedliche Geschlechter, darunter auch weiblich, einbringen würde, wäre es dir egal wie der Charakter dann aussieht? Oder ist es eher wichtig, dass der Charakter auch ähnlich wie du aussieht?
    \item \interview{P2} Der Charakter muss nicht wie ich aussehen. Da reicht es mir auch, wenn der Charakter weibliche Züge hat. Aber persönlich fände ich es noch besser, wenn man den Charakter auch selber anpassen könnte. 
    \item \interview{P2} Das Minispiel beim Training ist zu schwer.
    \item \interview{IV} Würdest du dir wünschen, dass es langsamer wäre?
    \item \interview{P2} Ja. Das auf alle Fälle.
    \item \interview{P2} Ich glaube er hat Hunger. Aber ich habe gerade keine Lust, ich will zuerst erkunden. Ich hoffe nur, dass er dadurch nicht sterben kann.
    \item \interview{DW} (Digimon defäkiert)
    \item \interview{P2} So etwas finde ich jetzt nicht so gut im Spiel.
    \item \interview{IV} Aus welchem Grund? Findest du das zu vulgär?
    \item \interview{P2} Ja. Super schlimm jetzt nicht, weil es ist ja auch ein Haustier. Aber ich mag eher eine süße Spielwelt und da hat so etwas nicht viel zu suchen.
    \item \interview{IV} Hängt das damit zusammen, dass du sagen würdest, dass es nicht für deine Zielgruppe ausgelegt ist? 
    \item \interview{P2} Hm. Wie war das denn bei Nintendogs zum Beispiel? Das habe ich mal gespielt als ich jünger war, aber ich erinnere mich nicht mehr genau daran.
    \item \interview{IV} In Nintendogs hat der Hund sein Geschäft auch auf der Straße verrichtet. Man musste das ganze dann in eine Tüte einpacken.
    \item \interview{P2} Aber haben die das dann auch so richtig gezeigt?
    \item \interview{IV} Ja. Das war immer beim Spazierengehen.
    \item \interview{P2} Ah, ich erinnere mich sogar daran. Aber bei Nintendogs fande ich das irgendwie nicht so wirklich schlimm.
    \item \interview{DW} (Der erste Kampf beginnt)
    \item \interview{P2} Ich verstehe hier gar nichts. Irgendwie kann ich nicht wirklich etwas drücken, außer \frq  D. vertrauen\flq{}.
    \item \interview{P2} Ich bin mir nicht sicher ob ich in diesem Gebiet schon gewesen bin.
    \item \interview{IV} Was planst du aktuell?
    \item \interview{P2} Ich möchte Digimon für die Stadt suchen. Aber ich frag mich erstmal in der Stadt um.
    \item \interview{P2} Ich habe jetzt Fleisch erhalten. Mein Digimon hatte vorhin schon Hunger, aber jetzt hätte ich endlich etwas.
    \item \interview{P2} (Füttert Partner)
    \item \interview{DW} (Partner weigert sich)
    \item \interview{P2} Was? Der wollte das nicht essen?
    \item \interview{IV} Aus welchem Grund denkst du, dass er nicht wollte?
    \item \interview{P2} Er möchte wahrscheinlich nur essen was er auch symbolisiert. Aber da war ein Fleisch-Symbol und ich habe ihm auch Fleisch gegeben. Ich weiß nicht genau wo das Problem liegt.
    \item \interview{DW} (Partner symbolisiert Hunger)
    \item \interview{P2} Ich versuche es jetzt nochmal. 
    \item \interview{P2} (Füttert Partner)
    \item \interview{P2} Anscheinend hat es jetzt geklappt. Aber ich weiß noch nicht wieso er vorhin nichts essen wollte. 
    \item \interview{IV} Hast du alle Informationen verstanden die dir der Stadtbewohner mitgegeben hat?
    \item \interview{P2} Ja. Ich soll mein Digimon pünktlich schlafen legen und ich sehe auch gerade, dass wir Nacht haben. Allerdings weiß ich noch nicht wo ich ihn schlafen legen kann. 
    \item \interview{P2} Achso. Hier im Menü ist die Option ihn schlafen zu legen. Anscheinend kann er auch überall schlafen.
    \item \interview{IV} Woher weißt du, dass du dieses Digimon zwei mal ansprechen musst?
    \item \interview{P2} Puh. Das ist eine extrem gute Frage. Ich denke, dass das einfach intuitiv ist. Aber es könnte sein, dass ein neuer Spieler das nicht machen würde.
    \item \interview{P2} Jetzt muss er auf die Toilette. Zum Glück weiß ich bereits wo die ist.
    \item \interview{IV} Woher genau?
    \item \interview{P2} Direkt in der Anfangsstadt war eine Toilette. Da ist eine japanisches Squat-Toilette. Ich kann mir vorstellen, dass viele im westlichen Markt das nicht gesehen hätten. Aber ich beschäftige mich in meiner Freizeit gerne mit Japan und lerne auch gerade japanisch.
    \item \interview{DW} (Ein weiterer Kampf beginnt)
    \item \interview{P2} Hier war doch vorhin ein anderes Digimon.
    \item \interview{DW} (Das Digimon ist besiegt und fällt zu Boden)
    \item \interview{P2} Das ist schon irgendwie brutal.
    \item \interview{IV} Aus welchem Grund findest du das brutal?
    \item \interview{DW} (Das Digimon steht auf und rennt davon)
    \item \interview{P2} Ach, egal. Wenn er am Ende überlebt, dann finde ich das doch gar nicht so schlimm. Ich habe nur vergessen wie das beim ersten Kampf gewesen ist. 
    \item \interview{DW} (Ein weiterer Kampf beginnt)
    \item \interview{DW} (Partner wurde besiegt)
    \item \interview{P2} Mir wurde gesagt, dass meine Items geklaut wurden. Aber ich habe jetzt nicht auswendig gelernt welche ich vorher hatte. Ich finde es auch doof, dass ich immer das Menü öffnen muss, damit ich nachschauen kann wie viel Leben ich noch habe. 
    \item \interview{IV} Hättest du das lieber dauerhaft angezeigt bekommen?
    \item \interview{P2} Ja. 
    \item \interview{IV} Aus welchem Grund gehst du nun ins Trainingscamp?
    \item \interview{P2} Ich war vorhin im Kampf und habe verloren, weil ich zu wenig Leben hatte. Deswegen wollte ich jetzt Leben trainieren.
    \item \interview{IV} Was planst du gerade?
    \item \interview{P2} Ich würde gerne einen anderen Weg gehen. Aber ich finde leider keinen. Ich laufe jetzt glaube ich zum zweiten Mal im Kreis.
    \item \interview{IV} Hättest du einen Ansatz für eine Lösung dieses Problems?
    \item \interview{P2} Eine Karte fände ich nicht verkehrt.
    \item \interview{IV} In welcher Form? 
    \item \interview{P2} In Form einer Minimap. Die sollte man aber auch im Menü ganz anzeigen können.
    \item \interview{P2} Ich kann bei diesem Kampf nicht abhauen.
    \item \interview{IV} Aus welchem Grund denkst du ist das so?
    \item \interview{P2} Vielleicht, weil dieser Kampf für die Story relevant ist.
    \item \interview{P2} Das Digimon hat gerade gesagt, dass ich das Geröll wegschleppen kann. Aber ich sehe das leider nicht.
    \item \interview{P2} Jetzt habe ich es gefunden.
    \item \interview{IV} Wo genau lag die Schwierigkeit oder was hättest du anders gestaltet?
    \item \interview{P2} Der Geröllhaufen sah eher aus wie Hintergrund. Das hätte man hervorheben sollen.
    \item \interview{IV} Würdest du sagen, dass eine emotionale Bindung zu deinem Digimon besteht? Also würdest du sagen, dass dein Digimon etwas ist worum du dich kümmern musst?
    \item \interview{P2} Ja, schon. Ich muss mich schon darum kümmern. Es kann nicht ohne mich. Es ist von mir abhängig.
    \item \interview{IV} Würdest du denn auch sagen, dass eine emotionale Bindung besteht? Also siehst du dein Digimon als Partner?
    \item \interview{P2} Gerade im Moment würde ich ihn noch austauschen gegen ein Digimon, welches mir richtig gut gefällt. 
    \item \interview{IV} Also magst du dein Partner gar nicht?
    \item \interview{P2} Ich finde andere süßer. Ich würde meinen auch gerne wechseln.
    \item \interview{P2} Langsam habe ich auch kein Fleisch mehr. Aber ich weiß nicht mehr wo ich das herbekomme. Ich such die Person mal, aber ich weiß leider nicht mehr genau wo.
    \item \interview{IV} Du hattest vorher erwähnt, dass du eine Karte in dem Spiel mögen würdest. Kannst du dir eine Lösung für das Problem mit einer Karte vorstellen?
    \item \interview{P2} Ja. Einfach ein Essens-Symbol auf die Karte packen. Dann weiß ich immer wo ich etwas bekomme.
    \item \interview{IV} Was ist aktuell dein Plan?
    \item \interview{P2} Etwas anderes als den Bergbau erkunden. Das ist eine ziemlich große Welt. Falls die noch größer ist müsste man immer viel laufen. Vielleicht kann man irgendwo teleportieren.
    \item \interview{IV} Aus welchem Grund meidest du gerade kämpfe?
    \item \interview{P2} Ich habe kaum Leben, keine Heildisks und weiß leider nicht ganz wo ich bin. Ich würde gerne in die Stadt, damit ich heilen kann, aber ich weiß leider nicht mehr wo ich hin soll. Die Fische haben mich in den Dschungel getragen und jetzt weiß ich gar nichts mehr. Ich weiß leider auch nicht mehr wo eine Toilette ist. Vielleicht war hier eine in der Nähe, aber ich finde keine. So etwas könnte man auch auf der Map aufnehmen.
    \item \interview{DW} (Ein Kampf beginnt)
    \item \interview{IV} Findest du, dass das Kampfsystem im Spiel so bleiben sollte oder würdest du dir etwas anderes wünschen?
    \item \interview{P2} Das kommt ganz darauf an. Also an sich fände ich es noch cool, falls man bestimmte Tasten rhythmisch drücken müsste. 
    \item \interview{DW} \frq  Alarm! Eindringlinge! Ich schieße!\flq{}
    \item \interview{P2} Ich hole mir erstmal die Items.
    \item \interview{DW} (Centauromon schießt)
    \item \interview{P2} (Prüft das Status-Menü) 
    \item \interview{P2} Der hat nur noch ein Leben. Anscheinend darf ich hier nicht lang.
    \item \interview{P2} Ich heile ihn kurz, dann probiere ich das nochmal. Soll ich jetzt zum Kreis oder Kreuz laufen?
    \item \interview{DW} (Centauromon schießt)
    \item \interview{P2} Jetzt habe ich keine Heildisk mehr.
    \item \interview{IV} Aus welchem Grund denkst du, dass du die Situation nicht ganz meistern konntest?
    \item \interview{P2} Ich denke mir hat das Verständnis für das Minispiel gefehlt. Ich habe nicht ganz verstanden was ich machen soll und wann geschossen wird.
    \item \interview{IV} Würdest du sagen, dass dir auch an anderen Stellen das Verständnis fehlt? 
    \item \interview{P2} Manchmal kann ich nicht ganz einschätzen wie stark Gegner sind.
    \item \interview{IV} Würdest du da gerne gewarnt sein, falls diese wesentlich stärker sind oder findest du das gut wie es ist?
    \item \interview{P2} Eine kleine Warnung wäre super, falls jemand wirklich viel zu stark ist. 
    \item \interview{IV} Würdest du sagen, dass du halbwegs erkennen kannst, welche Digimon stark sind und welche nicht?
    \item \interview{P2} Bis jetzt würde ich nur sagen, dass die Größe eine Rolle spielt. Groß würde in meinen Augen gleich mit stark gesetzt sein.
    \item \interview{P2} Es gibt etwas was ich noch nicht ganz verstanden habe, wenn das Digimon auf die Toilette geht. Ich bin mir nicht sicher ob die Freude des Digimons steigt oder das egal ist. Ich versteh die Leiste unten auch nicht ganz.
    \item \interview{IV} Was würde dir denn helfen die zu verstehen?
    \item \interview{P2} Wenn sie leer ist, dass das Digimon traurig ist und wenn sie voll ist, dass das Digimon glücklich ist.
    \item \interview{P2} (Digimon entwickelt sich)
    \item \interview{P2} Was? Ist das jetzt mein Digimon? Ich finde das mega doof. Ich möchte das alte Digimon zurück.
    \item \interview{IV} Wie findest du das, dass dein Partner sich verändert hat?
    \item \interview{P2} Ich finde das extrem doof. Er hat nichts mehr mit dem vorherigen Digimon zu tun. Das wäre für mich ein Grund das Spiel nicht mehr zu spielen. Das Digimon sieht echt doof aus.
    \item \interview{IV} Würdest du das auch sagen, wenn du einen anderen Partner erhalten hättest?
    \item \interview{P2} Mir ist es wichtig, dass es entweder süß, freundlich oder wenigstens cool aussieht. Ich will mich nicht um so ein ekliges Ding kümmern. Also am besten wäre es, wenn auch ich noch cooler aussehen würde. Dann wären wir ein Dream-Team. Aber so finde ich das doof. 
    \item \interview{IV} Also du findest die Design-Entscheidung von diesem Digimon zu vulgär. Sehe ich das richtig?
    \item \interview{P2} Ja.
    \item \interview{IV} Würdest du dir wünschen, dass dieses Digimon gar nicht im Spiel wäre oder ist es in Ordnung, wenn es als Gegner im Spiel implementiert ist?
    \item \interview{P2} Als Gegner noch in Ordnung, aber ich finde, dass der Bezug zum Partner einfach verloren geht.
    \item {-------------------} Abschlussfragen {-------------------}
    \item \interview{IV} Was findest du allgemein gut in Digimon World?
    \item \interview{P2} Es gibt nichts was ich überragend gut fand. Ich finde das ganze Spiel in Ordnung, aber nichts hat mich jetzt besonders vom Hocker gehauen.
    \item \interview{IV} Was fandest du schlecht in Digimon World?
    \item \interview{P2} Oftmals waren Forderungen vom Digimon, also die Sprechblasen, viel zu spät da. Meistens konnte man nicht mehr bis zur Toilette laufen.
    \item \interview{IV} Wie hättest du dieses Problem gelöst?
    \item \interview{P2} Fortschrittsbalken hätten schon ausgereicht.
    \item \interview{P2} Ich mag auch nicht, dass man in dem Spiel wenig zum Charakter personalisieren kann. Ebenfalls hätte ich mir auch eine bessere Einleitung ins Spielgeschehen gewünscht.
    \item \interview{IV} Was hältst du vom Kampfsystem? Was fandest du gut?
    \item \interview{P2} Ich fande gut, dass man den Partner im Kampf heilen konnte. Ich fand den Einsatz der Items gut umgesetzt.
    \item \interview{IV} Was fandest du nicht so gut umgesetzt?
    \item \interview{P2} Ich fände es gut, wenn man starken Angriffen ausweichen könnte. Mich würde es ebenfalls freuen, wenn man mehr Aktionen durchführen könnte die das Kampfgeschehen auch beeinflussen.
    \item \interview{IV} Du hattest ja vorher erwähnt, dass du Rhythmus-Elemente in dem Spiel interessant fändest. Würde das dem, von dir beschriebenen, Problem entgegenwirken?
    \item \interview{P2} Ja. Das könnte das ganze etwas spannender machen.
\end{itemize}}
\nolinenumbers