\newcommand{\interview}[1]{[#1]} 
\section{Transkript - Person 1}
\sloppy
\texttt{\begin{itemize}[]
    \setlength\itemsep{0.02em}
    \linenumbers
    \item \interview{IV} Wie alt bist du?
    \item \interview{P1} Ich bin 23 Jahre alt.
    \item \interview{IV} Welchem Geschlecht fühlst du dich zugehörig?
    \item \interview{P1} Männlich.
    \item \interview{IV} Fühlst du dich für das Interview ausreichend vorbereitet?
    \item \interview{P1} Ja.
    \item \interview{IV} Dann möchte ich dir zunächst ein paar Eingangsfragen stellen.
    Welche Spiele hast du in letzter Zeit gespielt?
    \item \interview{P1} In den letzten Wochen hauptsächlich Rocket League und Beat Saber.
    \item \interview{IV} Welche Genres spielst du allgemein gerne?
    \item \interview{P1} Recht viele, aber ich würde allgemein sagen, dass ich eher keine älteren Spiele spiele. Vielleicht so etwas wie Pokémon, aber das habe ich auch schon Ewigkeiten nicht mehr gespielt. Ich würde vielleicht noch hinzufügen, dass ich nicht unbedingt Shooter spiele, aber auch da gibt es vereinzelt Ausnahmen.
    \item \interview{IV} Spielst du Spiele lieber mit Maus und Tastatur oder mit einem Controller?
    \item \interview{P1} Das hängt ganz vom Spiel ab. Spiele die mit Controller gut funktionieren spiele ich gern mit Controller und dasselbe andersherum.
    \item \interview{IV} Wie wäre deine Entscheidung, falls du diese abwägen müsstest?
    \item \interview{P1} Insgesamt würde ich sagen, dass ich Tastatur mehr präferiere. Allerdings handelt es sich hier auch nur um einen kleinen Prozentsatz. Ich wechsel das auch ziemlich gerne, falls ein Spiel das gut unterstützt.
    \item \interview{IV} Nutzt du lieber eine Konsole oder einen Computer zum spielen?
    \item \interview{P1} PC natürlich. PC Master Race!
    \item \interview{IV} Hattest du damals ein Tamagotchi oder Vergleichbares?
    \item \interview{P1} Nein.
    \item \interview{IV} Ist dir das Digimon-Franchise bekannt?
    \item \interview{P1} Ich hatte bisher keine Berührungspunkte damit.
    \item {-------------------} Spiel beginnt {-------------------}
    \item \interview{DW} \frq Hast du ein Digivice?\flq{}
    \item \interview{P1} Nein. 
    \item \interview{DW} \frq Nächste Frage. Was magst du lieber: Tag oder Nacht?\flq{} 
    \item \interview{P1} Ich wähle mal Nacht.
    \item \interview{DW} (Intro-Videosequenz startet)
    \item \interview{P1} Oha! Das Spiel sieht auf alle Fälle alt aus.
    \item \interview{P1} Obwohl? Dafür, dass das Spiel so alt ist sieht das Intro doch eigentlich ganz gut aus. Aber gucken wir mal wie das Spiel aussieht, falls ich in der Story drinnen bin. 
    \item \interview{DW} (Jijimon schickt den Spieler aus dem Haus)
    \item \interview{P1} Okay. Man ist echt ziemlich auf sich allein gestellt. Ich weiß noch nicht so genau was ich tun soll. Ich lauf erstmal nach unten und erkunde ein wenig die Gegend.
    \item \interview{P1} Hm. Gibt es irgendwo eine Map?
    \item \interview{P1} Okay, bisschen komisch, dass man mit der Dreieck-Taste zurückkommt. 
    \item \interview{DW} (Der erste Kampf beginnt)
    \item \interview{P1} Muss ich irgendwas machen? Also keine Ahnung ob ich irgendwas anderes auswählen kann. Ich würde ungerne den Kampf verlassen.
    \item \interview{P1} (Probiert alle Tasten aus und öffnet das Inventar)
    \item \interview{P1} Oh! Hier ist ein Inventar. Da könnte ich vermutlich irgendein Item auswählen, falls ich eins zur Verfügung habe.
    \item \interview{IV} Was denkst du aktuell beim Kampf gegen das andere Digimon?
    \item \interview{P1} Ich bin mir aktuell nicht sicher ob die Option \frq  D. vertrauen\flq{}  etwas bewirkt. Ich habe aktuell nämlich nicht das Gefühl.
    \item \interview{DW} (Der Kampf ist vorbei und das Digimon droppt eine Floppy-Disk als Item)
    \item \interview{P1} Oh, dann gehe ich davon aus, dass man so speichern kann?
    \item \interview{IV} Wie kommst du darauf? Liegt es daran, dass dir die Floppy-Disk allgemein als Speicher-Symbol bekannt ist?
    \item \interview{P1} Genau. Ich hätte deswegen vermutet, dass das Spiel jetzt gespeichert hat.
    \item \interview{P1} Oh, aber wenn ich mir das Item im Inventar anschaue, dann steht hier \frq  kl. Heildisk\flq{}. Ich gehe also mal davon aus, dass die zum heilen meines Digimons gedacht ist.
    \item \interview{P1} (Findet ein Item)
    \item \interview{P1} Ein Pilz? Vielleicht ein giftiger Pilz. Ich probier das mal aus. Vielleicht heilt mein Digimon ja dadurch.
    \item \interview{P1} Oh hier auf dem Boden ist Wegweiser. Das wäre aber cooler, wenn ich alle Wege direkt lesen könnte statt auf die Animation zu warten. 
    \item \interview{DW} (Ein Digimon läuft auf den Spieler zu)
    \item \interview{P1} Oha. Nein, wir kämpfen nicht. 
    \item \interview{DW} (Der Kampf beginnt)
    \item \interview{P1} Okay, wir kämpfen anscheinend doch. 
    \item \interview{P1} (Ein weiterer Kampf beginnt)
    \item \interview{P1} Warte mal. Das ist doch das gleiche Digimon, oder nicht? Ich dachte, dass jedes einmal da ist und ich alle einsammeln soll. Ich weiß ja leider nicht wie viele Digimon es gibt. Kann ja sein, dass es unendlich viele Digimon gibt.
    \item \interview{DW} (Digimon symbolisiert, dass es auf Klo muss)
    \item \interview{P1} Jetzt muss er aufs Klo. Ja, dann mach doch!
    \item \interview{P1} Hmm. Wir gehen einfach zu den Typen am Anfang zurück. Vielleicht kann man ja in seinem Haus auf die Toilette. Aber wenn ich die Uhr richtig lese, dann sollte es bald sowieso Nacht sein.
    \item \interview{IV} Wie genau meinst du das?
    \item \interview{P1} Naja, könnte sein, dass die Uhr sich schon drei Mal im Kreis gedreht hat und ich nichts mitbekommen habe. 
    \item \interview{DW} (Digimon defäkiert außerhalb des sichtbaren Bereichs)
    \item \interview{P1} Hm. Ich weiß jetzt nicht ob er wirklich auf den Boden gemacht hat, aber er scheint nicht mehr zu müssen.
    \item \interview{P1} Oh. Hier in dem Laden ist das Digimon was in die Stadt ziehen wollte.
    \item \interview{P1} Aber wo war jetzt das Haus von dem Typen? Habe ich mich jetzt schon verlaufen?
    \item \interview{P1} (Erhält Starter-Items)
    \item \interview{P1} Oh guck mal. Ein Tutorial und ein paar Items. Bisschen komisch, aber ich glaube am Anfang hatte er auch gesagt, dass ich nochmal zurückkommen soll. Hätte ich da mal auf ihn gehört. 
    \item \interview{IV} Fändest du es besser, wenn man das anders gelöst hätte?
    \item \interview{P1} Ja. Auf alle Fälle. Sonst hätte ich diese Items am Anfang und das kurze Tutorial fast verpasst.
    \item \interview{IV} Findest du, dass es eine Eingangsfrage am Anfang geben sollte, ob der Spieler das Spiel und seine Mechaniken bereits kennt und basierend auf der Antwort überspringt man das Tutorial oder überspringt das Tutorial nicht?
    \item \interview{P1} Ja. Das fände ich um einiges besser. Dann könnten erfahrenere Spieler ein lästiges Tutorial überspringen und neue Spieler würden sich zuerst mit den Mechaniken vertraut machen bevor sie von allem überfordert wären. Es wäre zum Beispiel auch cool zu wissen wo eine Toilette ist. Das eine Digimon meinte zwar, dass ich mich darum kümmern muss, aber ich wüsste nicht wo genau.
    \item \interview{IV} Also würdest du dir wünschen, dass so etwas auch im Tutorial vorhanden wäre?
    \item \interview{P1} Ja. Ich fände es gut, wenn man wenigstens die Grundfunktionen erklären könnte. Ich wüsste jetzt nämlich spontan nicht wie genau man sich um das Digimon kümmern muss. Ich könnte ihm zum Beispiel noch mehr zu essen geben, aber ich weiß nicht ob er das möchte.
    \item \interview{P1} Hier steht \frq  Sättigt Digimon etwas\flq{} beim Fleisch. Ich finde, dass mir das nicht viel Informationen gibt.
    \item \interview{IV} Was würdest du dir denn wünschen, damit das deinen Vorstellungen entspricht?
    \item \interview{P1} Ich fände konkretere Werte wären wesentlich interessanter. So wie es auch bei der \frq  kl. Heildisk\flq{} ist.
    \item \interview{P1} Nach und nach findet man sich ins Spiel ein und versteht das immer ein wenig mehr. Ich habe jetzt das Gefühl, dass ich die Uhr auch verstanden habe.
    \item \interview{IV} Du hattest ja vorher erwähnt, dass du Schwierigkeiten damit hattest. Fändest du eine digitale Uhr besser wäre?
    \item \interview{P1} Das weiß ich nicht so genau. Es mag sein, dass die digitale Uhr verständlicher wäre. Allerdings finde ich es gut, dass man hier nicht zwangsweise die Zeit in irgendwelche realen Nummern pressen möchte. Dadurch fühlt sich die Welt für mich realer an. Hier wird nicht angezeigt, dass das jetzt eine Minute oder Stunde ist, sondern irgendwann ist es einfach Tag und irgendwann Nacht. 
    \item \interview{IV} Würdest du sagen, dass du die angezeigte Oberfläche verstanden hast?
    \item \interview{P1} Ich würde schon sagen, dass ich die verstanden habe. Oben ist die Uhr abgebildet und die beiden Leisten zeigen einmal an wie glücklich das Digimon ist und wie müde das Digimon ist.
    \item \interview{IV} Wie kommst du zu diesem Schluss?
    \item \interview{P1} Das lächelnde Emoticon lässt mich darauf schließen, dass es um die Fröhlichkeit meines Digimons geht. Bei dem anderen Balken bin ich mir nicht komplett sicher, aber das ist einer der wenigen Werte über die im Spiel gesprochen wurde und das hätte jetzt spontan für mich Sinn gemacht. 
    \item \interview{P1} Bin ich schon an diesem Ort gewesen?
    \item \interview{IV} Findest du die Spielwelt unübersichtlich?
    \item \interview{P1} Ich denke, dass mich gleich noch ganz gut zurechtfinden werden. Aber wenn das alles noch viel größer wird, dann finde ich, dass eine Karte schon mit dabei sein sollte.
    \item \interview{IV} In welcher Form?
    \item \interview{P1} Das weiß ich noch nicht. Ich probier jetzt einmal das normale Training und einmal das Bonus-Training aus. 
    \item \interview{P1} Dann einmal den Bonus-Versuch. Aber ich schätze mal, dass er dadurch schneller hungrig wird.
    \item \interview{P1} (Probiert den Bonus-Versuch aus)
    \item \interview{P1} Oh warte mal. Das unten ist vielleicht doch nicht die Müdigkeit.
    \item \interview{IV} Wie kommst du jetzt darauf oder was genau macht dich jetzt stutzig?
    \item \interview{P1} Also das blaue ist jetzt voll. Mehr oder wenig. Obwohl? Keine Ahnung, irgendwie ist das jetzt auch rot. Ich weiß nicht genau was voll und was leer ist. 
    \item \interview{IV} Also würdest du sagen, dass du ein Problem damit hast die Balken zu verstehen?
    \item \interview{P1} Ja genau. Da weiß ich jetzt nicht wirklich ob das voll oder leer ist. Das fällt mir schwer abzuschätzen. 
    \item \interview{P1} Ich konnte beim Kampf keinen wirklichen Sinn für den \frq  D. vertrauen\flq{} feststellen. Also vielleicht lenke ich den Gegner irgendwie ab, wenn ich das öfters drücke?
    \item \interview{IV} Wüsstest du wo du dich heilen kannst, falls du im Kampf Schaden genommen hast?
    \item \interview{P1} Also ich weiß, dass ich mich mit der \frq  kl. Heildisk\flq{} heilen könnte und wenn man im Haus von diesem alten Typen ruht. Aber ich denke auch, wenn man schläft heilt man sich.
    \item \interview{IV} Auf die Vermutung kommst du, weil du in der Stadt ruhen kannst und dich das heilt?
    \item \interview{P1} Ja genau. Ich denke, dass schlafen allgemein heilt. Also das habe ich jetzt nicht extra nachgeguckt, aber ich glaube ich habe letztes mal geheilt als ich geschlafen habe. Ich sehe gerade aber im Digimon Menü, dass diese Leiste anscheinend Disziplin bedeutet. Hier steht auch, dass er insgesamt drei Leben hat. Okay. Das ergibt jetzt alles Sinn, aber irgendwie ist das alles nicht so intuitiv.
    \item \interview{P1} Ich glaube ich habe jetzt eine Toilette gefunden.
    \item \interview{DW} (Ein Digimon läuft auf den Spieler zu)
    \item \interview{P1} Ein weiterer Kampf? Das ist sogar ein Doppelkampf? Gegen zwei Digimon werde ich das nicht schaffen. Ich flüchte lieber mal. 
    \item \interview{P1} (Flüchtet aus dem Kampf)
    \item \interview{P1} Ich würde jetzt nur gerne mein Digimon ermutigen, dass es nicht schlimm war in diesem unfairen Kampf zu flüchten. Ich habe den zweiten ja nicht einmal ordentlich sehen können. Ich habe nur noch die Sprechblase im letzten Moment gesehen.
    \item \interview{IV} Also hättest du gerne eine weitere Option neben Loben und Tadeln?
    \item \interview{P1} Die müsste jetzt nicht unbedingt da sein, aber an sich wäre das schon cool. 
    \item \interview{P1} Ich weiß nur nicht ob ich überhaupt irgendwo Essen sehe.
    \item \interview{IV} Wie meinst du das?
    \item \interview{P1} Also eine Anzeige wie voll mein Digimon aktuell ist. Weil ich glaube, dass man das gar nicht sieht. 
    \item \interview{IV} Wie hast du die Essen-Mechanik bis jetzt verstanden?
    \item \interview{P1} Das Digimon hat entweder Hunger oder keinen Hunger.
    \item \interview{IV} Hättest du gerne eine Anzeige wie voll dein Digimon aktuell ist?
    \item \interview{P1} Also eigentlich finde ich das wie es aktuell ist okay. Aber ich bin allgemein ein Typ der gerne alle möglichen Arten von Informationen sieht. Aber ich denke auch nicht, dass es unbedingt negativ für das Spielerlebnis ist. 
    \item \interview{P1} Ich habe das Gefühl, dass die Leisten sich allgemein sehr langsam ändern. Das passiert in anderen Spielen viel schneller.
    \item \interview{IV} Hast du ein Beispielspiel wo das der Fall ist?
    \item \interview{P1} Ich habe ein Spiel im Kopf, aber mir fällt der Name nicht ein. Also man hat auch Tageszyklen und braucht dann Aktionen auf. Mir fällt der Name nicht ein, aber Stardew Valley hat so etwas ähnliches.
    \item \interview{IV} Aus welchem Grund hast du die Kämpfe aktuell gemieden?
    \item \interview{P1} Ich wollte unbedingt zu den Ozean den die immer erwähnt haben.
    \item \interview{IV} Und du wolltest nicht unterbrochen werden auf deiner Reise?
    \item \interview{P1} Ja genau, aber manchmal bin ich mir nicht sicher welche Kämpfe ich wahrnehmen soll und welche nicht. Weil die müssen ja auch zur Stadt kommen.
    \item \interview{IV} Also du hast immer noch die Hauptaufgabe dauerhaft im Hinterkopf, dass du Digimon in die Stadt einladen sollst?
    \item \interview{P1} Ja genau. Das auf jeden Fall.
    \item \interview{IV} Und das mithilfe von Kämpfen, weil du vorher die Erfahrung gemacht hast, dass die nach Kämpfen in die Stadt kommen?
    \item \interview{P1} Ja. 
    \item \interview{P1} Hier sind aber allgemein zu viele Gegner.
    \item \interview{IV} Würdest du sagen, dass du das alles als gefährlich einstufst?
    \item \interview{P1} Ja, da das nur das Anfangsgebiet des Spiels ist.
    \item \interview{P1} Och mann, jetzt will er wieder so viel essen.
    \item \interview{IV} Reichen deine Ressourcen dafür aus?
    \item \interview{P1} Ich würde sagen, dass das schon ausreicht. Man kriegt ja täglich ein paar Rationen und kann auch im Wald noch Pilze sammeln. 
    \item \interview{P1} Ich probier jetzt einfach mal die Bonus-Versuche beim Training aus. Aber ich weiß nicht wie viel Zeit vergeht dabei.
    \item \interview{IV} Weißt du denn wie viel Zeit bei den normalen Versuchen vergeht?
    \item \interview{P1} Bei den normalen Versuchen ist die Zeit immer ein deutliches Stück gesprungen, aber ich habe nicht das Gefühl, dass das bei den Bonus-Versuchen passiert ist.
    \item \interview{IV} Hättest du gerne einen Indikator dafür? Zum Beispiel eine Animation die die Uhr kurz ein Stück nach vorne bewegt?
    \item \interview{P1} Das wäre tatsächlich ziemlich cool. Gerade beim Training würde das glaube ich sehr viel helfen. 
    \item \interview{IV} Lohnen sich Kämpfe oder das Training mehr für dich?
    \item \interview{P1} Ich denke, dass sich das Training mehr lohnt. Also mit den Bonus-Versuchen vielleicht nicht, aber das normale Training sollte sich mehr lohnen.
    \item \interview{IV} Trainierst du jetzt nur noch, damit du die Zeit bis zum Anbruch der Nacht totschlagen kannst? 
    \item \interview{P1} Ja genau. Das ist aktuell mein Plan. Ich möchte nämlich so nah wie möglich noch am Dorf bleiben. Auch, wenn ich draußen auch schlafen kann. 
    \item \interview{P1} Mein Digimon ist aktuell ausgelaugt. Vielleicht darf er nicht den ganzen Tag trainieren.
    \item \interview{IV} Wie kommst du darauf?
    \item \interview{P1} Mir wird aktuell das Symbol für einen Tropfen angezeigt und ich hätte jetzt darauf geschlossen, dass das ein Schweißtropfen sein soll.
    \item \interview{IV} Würdest du sagen, dass du nach der einen Stunde Spielzeit jetzt bereits emotionale Bindungen zu deinem Digimon aufgebaut hast?
    \item \interview{P1} Ich habe schon das Gefühl. Dadurch, dass man die ganzen Statistiken, also wie glücklich das Digimon oder andere Bedürfnisse, hat gibt das einem schon das Gefühl, dass eine Bindung besteht. 
    \item \interview{IV} Würdest du sagen, dass es in deinem Sinne ist, dass es ihm so gut geht wie möglich? Oder würdest du sagen, dass es nicht schlimm wäre, falls dein Partner mal einen Kampf verliert?
    \item \interview{P1} Also Verlieren oder ähnliches jetzt nicht, aber das würde ich auch nicht in anderen Spielen wollen. 
    \item \interview{IV} Würdest du sagen, dass das Spiel immersiv ist? Also könnte es auch passieren, dass du mal die Zeit vergisst?
    \item \interview{P1} Ja, ich denke schon, dass das passieren könnte. Dadurch, dass man viel erkunden kann denke ich schon, dass es sehr immersiv ist. Ich würde bei meinem Spielstil eine Mischung aus Digimon suchen und Welt erkunden machen.
    \item \interview{P1} Oh. Mein Digimon ist krank geworden.
    \item \interview{P1} (Versorgt Partner Medizin)
    \item \interview{DW} (Partner weigert sich)
    \item \interview{P1} Irgendwie möchte er nicht.
    \item \interview{IV} Was denkst du wieso er das nicht möchte, obwohl das in der Beschreibung des Items steht?
    \item \interview{P1} Ich könnte mir das so erklären, dass Digimon wie Tiere sind. Bei unseren Katzen ist das auch so, dass die nicht einfach so Medizin essen würden. Man muss das ganze dann ja ins Katzenfutter beimischen, damit die Katzen die Medizin einnehmen. 
    \item \interview{IV} Denkst du, dass dies im Spiel möglich ist?
    \item \interview{P1} Das weiß ich nicht.
    \item \interview{IV} Aber du würdest es vermuten, richtig?
    \item \interview{P1} Ja. Für mich persönlich und aufgrund meiner Erfahrung mit Tieren wäre dies das intuitivste.
    \item \interview{P1} Ich glaube aber, dass ich jetzt einen Kampf bestreiten möchte.
    \item \interview{IV} Was planst du mit der Krankheit deines Digimons zu machen?
    \item \interview{P1} Ich lass es erst einmal krank bleiben. Es bringt ja auch nichts jetzt doof rumzustehen und nichts zu machen. Das wird sich schon irgendwie ergeben.
    \item \interview{IV} Was denkst du ist das Resultat der Krankheit, falls du diese nicht heilst?
    \item \interview{P1} Ich denke, dass mein Digimon ein wenig langsamer im Kampf wird oder ein wenig Leben verliert. Aber ich würde jetzt nicht vermuten, dass er dadurch sterben kann. Was ich nochmal versuchen würde ist, dass ich das Digimon schlafen schicke. Vielleicht geht die Krankheit dann von alleine weg. 
    \item \interview{P1} Vielleicht muss er Hunger haben? Leider bringt das auch nichts. Aber vielleicht ist das ja wie im echten Leben und das kuriert sich aus.
    \item \interview{P1} (Spricht NPCs an)
    \item \interview{P1} Ich weiß nicht, ob ich das vorher schon einmal gelesen habe.
    \item \interview{DW} (Digimon digitiert)
    \item \interview{P1} Oh er ändert sich. Ich fand ihn vorher aber schöner. 
    \item \interview{P1} Die Werte sind jetzt bei 1400.
    \item \interview{IV} Also hast du bemerkt, dass die Werte auch gestiegen sind nach einer Digitation?
    \item \interview{P1} Ja, aber das sind denke ich auch Erfahrung von anderen Spielen. Da ist das meistens so. Ich denke, falls man gar keine Spielerfahrung hat könnte das problematisch werden. Ich weiß nicht ob man direkt darauf schließen könnte, dass die Werte sich erhöhen.
    \item \interview{IV} Du hast vermutlich bemerkt, dass die Krankheit nach der Digitation verschwunden ist. Hast du eine Idee wieso das passiert ist?
    \item \interview{P1} Ich denke mal, dass die Werte und Probleme beim digitieren überschrieben wurden.
    \item \interview{P1} (Läuft in ein Digimon rein)
    \item \interview{P1} Ich wollte nicht unbedingt kämpfen, aber okay. Dann nehme ich den Kampf jetzt mit.
    \item \interview{IV} Aus welchem Grund ist der Kampf jetzt passiert?
    \item \interview{P1} Das gegnerische Digimon war außerhalb der Kamera und deswegen bin ich reingelaufen. 
    \item \interview{IV} Wie findest du das?
    \item \interview{P1} Ich finde das nicht schlimm, da auch mal so ein Überraschungsangriff in Ordnung ist.
    \item \interview{P1} Ich habe nach der Digitation auch andere Kampftechniken erhalten.
    \item \interview{P1} Bei dem Kampf mit Kunemon ist das ein bisschen komisch, dass die Kamera so ein kleines Kampffeld erstellt. Ich habe auch eine weitere Kampfoption erhalten. Die bewirkt anscheinend, dass mein Digimon aggressiver angreift. Vielleicht hat dies aber auch negative Auswirkungen auf seine Fröhlichkeit. 
    \item \interview{P1} Kann man auch mit dieser Brücke interagieren? Anscheinend nicht. Das fände ich auch cool, wenn es mehr Optionen gäbe mit Dingen zu interagieren.
    \item \interview{IV} Du hättest also gerne mehr Kommentare vom Hauptcharakter zur Umgebung?
    \item \interview{P1} Genau.
    \item \interview{IV} Wenn man etwas findet, dass man auch damit belohnt wird Informationen dazu zu erhalten?
    \item \interview{P1} Ja, genau. Dann hat man nicht unbedingt etwas erhalten, aber wenigstens könnten die Informationen weiterhelfen.
    \item \interview{IV} Jetzt wo du einige Zeit mit dem neuen Digimon verbracht hast: Würdest du sagen, dass deine Beziehung sich nach der Digitation zum Partner verändert hat?
    \item \interview{P1} Ich würde schon sagen, dass die Beziehung definitiv anders ist.
    \item \interview{IV} Woran denkst du liegt das?
    \item \interview{P1} Ich denke an der Größe. Vorher sah der Partner eher aus wie ein Haustier, aber mittlerweile ist es so groß wie ein Elefant im Zoo. 
    \item \interview{IV} Kennst du ein vergleichbares Spiel wie Digimon World?
    \item \interview{P1} Ein vergleichbares leider nicht. Außer du zählst so etwas wie Nintendogs dazu. Aber da fehlen die Kämpfe.
    \item \interview{P1} Ich finde, dass man generell keine Aktionen unterdrücken können sollte. Oftmals ist es im Training so, dass ich trainiere, obwohl mein Partner Hunger hat. Sowas sollte nicht sein. 
    \item \interview{IV} Wie stellst du dir eine mögliche Lösung für dieses Problem vor?
    \item \interview{P1} Es würde ausreichen, wenn man daran gehindert wird zu trainieren und man sich zuerst um das Digimon kümmern sollte. 
    \item \interview{IV} Findest du dich von der Zeit im Spiel unter Druck gesetzt oder ist der Zeitrahmen genau richtig?
    \item \interview{P1} Ich finde, dass das alles relativ gut balanciert ist. Man hat eigentlich immer etwas zu tun und kann auch allen Orten schlafen. Ich finde also, dass die Zeit richtig ist.
    \item \interview{P1} Ich verstehe die \frq  Fluch\flq{}-Anzeige im Status-Menü nicht. Die ist immer ein bisschen voll, aber irgendwie verändert das nichts am Spielfluss.
    \item \interview{P1} Ich würde gerne den MP-Chip ausprobieren. Der sollte die maximale MP anheben, aber irgendwie möchte er ihn nicht fressen. 
    \item \interview{P1} Bei den Entwicklungen ist nicht immer ganz ersichtlich. Es folgt nicht wie bei Pokémon einer Logik, sondern scheint eher willkürlich.
    \item \interview{DW} \frq  Alarm! Eindringlinge! Ich schieße!\flq{}% Centauromon 
    \item \interview{P1} Okay, ich warte erst einmal. Ich erkunde erst einmal etwas anderes.
    \item \interview{IV} Aus welchem Grund hast du Sorgen weiterzugehen?
    \item \interview{P1} Ich denke, dass es erst einmal genug anderes zu erkunden gibt. Ich hätte Sorgen vor Nachteilen.
    \item \interview{IV} Würdest du dich mit vollem Leben in das Gebiet trauen?
    \item \interview{P1} Ja, das schon eher. Mich interessiert es ja auch stark was hinter diesem Gebiet ist. Jetzt habe ich nur die Frage, ob ich schon einmal in diesem Gebiet war.
    \item \interview{DW} (Digimon stirbt im Kampf)
    \item \interview{P1} Vielleicht hätte ich den wiederbeleben können? Mit Items aus dem Inventar? Na gut, dadurch ist mein Digimon jetzt nicht glücklich. Ich würde aber vermuten, dass man diese drei Leben wieder aufladen kann. 
    \item \interview{P1} Das Digimon möchte die Chips nicht essen. 
    \item \interview{IV} Was denkst du woran das liegt?
    \item \interview{P1} Vielleicht, weil das Item nicht zum Digimon passt? Aber wenn er das nicht nehmen möchte, dann trainieren wir jetzt eine Runde. 
    \item \interview{IV} Ich habe jetzt beobachten können, dass du öfters die Funktion zum Loben genutzt hast. Aus welchem Grund hast du bist jetzt die Funktion zum Tadeln nicht verwendet?
    \item \interview{P1} Ich habe bis jetzt keinen Sinn darin gesehen, weil er nur das macht was ich ihm sage. Wenn ich ihm also was falsches sage, kann ich meinen Partner ja nicht anmeckern, dass er es falsch gemacht hat.
    \item \interview{P1} Die Bank in der Stadt ist ziemlich praktisch. Dann muss man sich nicht zu viel Gedanken um den Inventarplatz machen. 
    \item \interview{P1} So, ich möchte mich jetzt aber in das neue Gebiet begeben. Es ist jetzt auch Nacht, vielleicht schlafen die und ich kann durch? Oh Tatsache. Jetzt kam keine Nachricht, dass die schießen möchten.
    \item \interview{DW} (Hauptcharakter wird angeschossen und zurückgeschlagen)
    \item \interview{P1} Wow, das hätte ich nicht erwartet. Ich bin wieder am Anfang des Gebiets. Aber ich habe nicht das Gefühl, dass ich jetzt keinen Nachteil dadurch hätte. Ich selbst als Spieler habe ja kein Leben. 
    \item \interview{P1} (Prüft das Status-Menü)
    \item \interview{P1} Aber mein Digimon hat jetzt nur noch ein LP.
    \item \interview{P1} (Gibt Digimon \frq  kl. Heildisk\flq{})
    \item \interview{DW} (Digimon lehnt ab)
    \item \interview{P1} Sowas kann ich zum Beispiel nicht verstehen. Ich verstehe ja, wenn ein Chip nicht zu meinem Digimon passt oder die Medizin fehlerhaft ist, dass das nicht geht. Aber das Item hatte er vorher doch auch schon gegessen und ich konnte das im Kampf einsetzen.
    \item \interview{IV} Welche Parameter könntest du dir denn vorstellen, welche zu diesem Verhalten führen?
    \item \interview{P1} Ich denke mal, dass das Digimon nicht auf dich hört. Also, dass er dich nicht als Respektsperson sieht. Keine Ahnung. Aber da wüsste ich nicht wie ich das ausbauen sollen. 
    \item \interview{P1} Ah. Ich glaube ich habe es. Wenn er keine Disziplin hat, dann isst er das nicht. Also muss ich ihn vielleicht einfach nur Tadeln?
    \item \interview{P1} (Tadelt Digimon)
    \item \interview{P1} (Gibt Digimon \frq  kl. Heildisk\flq{})
    \item \interview{P1} Wow! Das funktioniert ja sogar wirklich. Endlich isst er die. Vielleicht war das auch das Problem weswegen er die Medizin und den Chip nicht essen wollte?
    \item {-------------------} Abschlussfragen {-------------------}
    \item \interview{IV} Was hast du im Spiel als gut empfunden?
    \item \interview{P1} Die Vielfalt. Also, dass man generell große Gebiete hat die man erkunden kann und überall ein paar Items findet. Ich fand auch gut, dass überall ein paar Digimon rumstehen. Also für mich das positivste ist die Erkundung.
    \item \interview{IV} Würdest du sagen, dass es genügend Informationen gab und du immer wusstest was du erkunden möchtest?
    \item \interview{P1} In der Stadt gab es ja dieses eine Digimon, welche ein paar Himmelsrichtungen genannt hat in die man gehen könnte. Ich denke schon, dass man dazu angeregt wurde sich das mal anzuschauen. Also ich denke, dass es da genügend Informationen gibt. 
    \item \interview{IV} Du hattest einmal die Schwierigkeit einer Himmelsrichtung zu folgen, weil gesagt wurde, dass sich etwas im Süden befindet und du es zunächst nicht gefunden hattest.
    \item \interview{P1} Genau, da hatte ich das ganze ja auch nur durch Zufall gefunden.
    \item \interview{IV} Fändest du eine Map mit einer groben Richtung besser, um solche Zufälle auszuschließen?
    \item \interview{P1} Ich weiß nicht. Ich glaube eine Map fändet ich eigentlich cool. Gerade wenn die Welt noch wesentlich größer wäre, dann könnte man das ja so machen, dass diese sich nach und nach aufdeckt.
    \item \interview{IV} Jetzt als Gegenfrage zu der ersten. Was war in deinem Empfinden nicht so gut umgesetzt?
    \item \interview{P1} Dass man kein wirkliches Ziel hat. Also am Anfang hat man gesagt bekommen, dass man so viele Digimon wie möglich finden soll und zur Stadt bringen soll und so weiter. Aber es war halt immer irgendwie ein bisschen zufällig, dass ich die gefunden habe.
    \item \interview{IV} Was hätte dir geholfen dieses Problem zu bewältigen?
    \item \interview{P1} Ich hätte gerne eine Aufgabenliste. Sonst läuft man nur zufällig durch die Gegend und findet manchmal Digimon, welche in die Stadt möchten. Das ist genauso wie mit dem Digimon das ich gefunden habe als ich zufällig in ein Gebiet gelaufen bin. Aber es hat sich nicht so angefühlt, als hätte ich eine Aufgabe erfüllt, weil das kein wirkliches Ziel war.
    \item \interview{IV} Denkst du, dass, zusätzlich zu der Aufgabenliste, eine Map zu der Bewältigung des Problems beitragen würde? Oder würde das eher dazu führen, dass du weniger erkundest und nur zum Ziel auf der Map läufst?
    \item \interview{P1} Weiß ich nicht. Ich denke, dass das stark darauf ankommt welche Art von Spieler man selbst ist. Bei mir persönlich wäre das so, dass ich trotzdem noch die Welt erkunden würde. Aber das liegt auch an meinem Spielstil. Ich würde aber auch sagen, dass ich manchmal den Spielstil wechsel. Du hattest ja vorhin schon gesehen, dass ich manchmal Kämpfe meide und was neues suche und manchmal auch die Kämpfe mitnehme. Aber das ist auch in anderen Spielen so.
    \item \interview{IV} Magst du das mit den anderen Spielen noch näher erläutern? Wonach genau suchst du da? Sind das Nebenmissionen oder andere Dinge?
    \item \interview{P1} Ja genau, meistens sind das auch Nebenmissionen nach denen ich suche oder Items. 
    \item \interview{IV} Was hältst du vom Kampfsystem? Was ist dir positiv herausgestochen?
    \item \interview{P1} Ob es jetzt wirklich was positives, was herausgestochen ist, gibt weiß ich nicht. Ich finde das sie von alleine kämpfen vielleicht gar nicht so schlecht. Wenn man jetzt die parallele zu Pokémon zieht. Da muss man ja jedes mal die Angriffe selbst auswählen. Aber ich weiß nicht ob ich das jetzt zwangsweise positiv sehen würde.
    \item \interview{IV} Und was würdest du als negativ sehen?
    \item \interview{P1} Das ich nicht wirklich was die ganzen Optionen bewirken. Also ich habe immer nur Vermutungen was die einzelnen Aktionen machen, aber ich bin mir da nicht sicher. 
    \item \interview{IV} Würdest du dir mehr Einfluss beim Kampf wünschen?
    \item \interview{P1} Also wenn das jetzt so ist, dass die einzelnen Aktionen nur Bewegungsmuster sind, dann finde ich das in anderen Spielen besser. Da wird so etwas deutlicher kommuniziert. Dann fände ich das vielleicht auch gar nicht so schlecht, wenn man einzelne Angriffe im Kampf kommandieren könnte. So, dass du ein Angriffsmuster und ein Verteidigungsmuster hast und diese noch mit einzelnen Kommandos überschreiben kannst. 
    \item \interview{IV} Dir scheint es also wichtig zu sein, dass man ein bisschen mehr Kontrolle über das Kampfgeschehen hat. In welchen Maßen sollte das erfolgen?
    \item \interview{P1} Mir ist es wichtig, dass das Digimon immer noch selber kämpft und man die Option hat auch mal nichts zu machen. So wie ich das teilweise gemacht habe. Das sollte, meiner Meinung nach, immer noch möglich sein. Bei Sims ist es zum Beispiel auch möglich, dass die Charaktere selber agieren, aber man deren Verhalten überschreiben kann.
    \item \interview{IV} Würdest du sagen, dass das Spiel noch von einem anderem Kampfsystem profitieren würde? Wir hatten ja vorhin Pokémon als Beispiel bei dem man Attacken selber auswählen kann. Fallen dir noch andere ein die passen würden?
    \item \interview{P1} Also ich glaube die Positionierung des Digimons sollte nicht veränderbar sein. Also der Hauptcharakter sollte immer noch am Rand stehen und das Digimon sollte sich frei bewegen. Das sollte dann abhängig von den Modi sein in dem es sich befindet. Aber ansonsten weiß ich es nicht so ganz. Was ich aber cool fände wäre, wenn man bei mehreren Gegnern auswählen könnte welchen man fokussieren möchte.
\end{itemize}}
\nolinenumbers